%\title{Project Report}
%
%%% Preamble
\documentclass[paper=a4, fontsize=11pt]{scrartcl}
\usepackage[T1]{fontenc}
\usepackage{fourier}

\usepackage[english]{babel}															% English language/hyphenation
\usepackage[protrusion=true,expansion=true]{microtype}	
\usepackage{amsmath,amsfonts,amsthm} % Math packages
\usepackage[pdftex]{graphicx}	
\usepackage{url}
\usepackage{hyperref}
\usepackage{graphicx}
\usepackage{wrapfig}
\usepackage[margin=0.75in]{geometry}

%%% Custom sectioning
\usepackage{sectsty}
\allsectionsfont{\centering \normalfont\scshape}


%%% Custom headers/footers (fancyhdr package)
\usepackage{fancyhdr}
\pagestyle{fancyplain}
\fancyhead{}											% No page header
\fancyfoot[L]{}											% Empty 
\fancyfoot[C]{}											% Empty
\fancyfoot[R]{\thepage}									% Pagenumbering
\renewcommand{\headrulewidth}{0pt}			% Remove header underlines
\renewcommand{\footrulewidth}{0pt}				% Remove footer underlines
\setlength{\headheight}{3.6pt}
\date{}


%%% Equation and float numbering
\numberwithin{equation}{section}		% Equationnumbering: section.eq#
\numberwithin{figure}{section}			% Figurenumbering: section.fig#
\numberwithin{table}{section}				% Tablenumbering: section.tab#


%%% Maketitle metadata
\newcommand{\horrule}[1]{\rule{\linewidth}{#1}} 	% Horizontal rule

\title{
		\vspace{-1in} 	
		\usefont{OT1}{bch}{b}{n}
		\normalfont \normalsize \textsc{Durham Computer Science} \\ [5pt]
		\horrule{0.5pt} \\[0.4cm]
		\huge  GPU, Many-core and Cluster Computing Assignment - LLLL76\\
		\horrule{2pt} \\[0.5cm]
		\vspace{-1in} 	
}

%%% Begin document
\begin{document}
\maketitle
\section{Step One}
Report demonstrates a mastery of performance analysis tools (5). Bottleneck of codes are identified and characterised (5). Performance model is set up in the report and it is calibrated to the used machine and code (5). Some reasonable predictions are made how much can be improved by performance optimisation (5).
\subsection{Results}

\subsection{Bottlenecks}

\subsection{Performance Model}

\subsection{Predictions}

\section{Step Two}
Report gives a clear description what has been realised in the code (5). Vectorisation report results are summarised in report and discussed (5). Runs with various input data sets have been made (5). Results are presented with state-of-the-art techniques (proper figures, e.g.) (5).
\subsection{Description}

In step two we work to split the computational domain into smaller cubes and vectorise any parts that can be. This can be done through reducing the branching at a showstopping point in the code. We start by splitting the overall box we are working in into smaller cubes, the dimensions of these cubes is hard-coded at a single point in the code, this is easy to change. If the dimensions of the box is such that it cannot be made up of complete cubes then the code will halt. If the input variables are such that the box can be made up of a whole number of cubes then the code will move onto working out if a given cube can be vectorised, this would be the case if, out of all the cells in the given cube, none of them are part of the obstacle. If the given block contains no cells from the obstacle then it can be vectorised as in the standard realisation of the code the if statement is a check if the current cell is part of the obstacle. Thus the pre-processing for checking the status of the cells is required to allow the non-obstacle blocks to be vectorised and the blocks that do contain part of obstacle to be treated with the standard code given, using the if statement.

The vectorisation is done using the standard intel SIMD pragma.

\subsection{Vectorisation Reports}

\subsection{Results}

\section{Step Three}
Report gives a clear description what has been realised in the code (5). Runs with various input data sets have been made (5). Results are interpreted and future work and shortcomings are identified (5). Results are presented with state-of-the-art techniques (proper figures, e.g.) (5).
\subsection{Description}

\subsection{Results}

\subsection{Limitations}

%%% End document
\end{document}